%% Generated by Sphinx.
\def\sphinxdocclass{report}
\documentclass[letterpaper,12pt,english]{sphinxmanual}
\ifdefined\pdfpxdimen
   \let\sphinxpxdimen\pdfpxdimen\else\newdimen\sphinxpxdimen
\fi \sphinxpxdimen=.75bp\relax
\ifdefined\pdfimageresolution
    \pdfimageresolution= \numexpr \dimexpr1in\relax/\sphinxpxdimen\relax
\fi
%% let collapsible pdf bookmarks panel have high depth per default
\PassOptionsToPackage{bookmarksdepth=5}{hyperref}

\PassOptionsToPackage{warn}{textcomp}
\usepackage[utf8]{inputenc}
\ifdefined\DeclareUnicodeCharacter
% support both utf8 and utf8x syntaxes
  \ifdefined\DeclareUnicodeCharacterAsOptional
    \def\sphinxDUC#1{\DeclareUnicodeCharacter{"#1}}
  \else
    \let\sphinxDUC\DeclareUnicodeCharacter
  \fi
  \sphinxDUC{00A0}{\nobreakspace}
  \sphinxDUC{2500}{\sphinxunichar{2500}}
  \sphinxDUC{2502}{\sphinxunichar{2502}}
  \sphinxDUC{2514}{\sphinxunichar{2514}}
  \sphinxDUC{251C}{\sphinxunichar{251C}}
  \sphinxDUC{2572}{\textbackslash}
\fi
\usepackage{cmap}
\usepackage[T1]{fontenc}
\usepackage{amsmath,amssymb,amstext}
\usepackage{babel}



\usepackage{tgtermes}
\usepackage{tgheros}
\renewcommand{\ttdefault}{txtt}



\usepackage[Bjarne]{fncychap}
\usepackage{sphinx}

\fvset{fontsize=auto}
\usepackage{geometry}


% Include hyperref last.
\usepackage{hyperref}
% Fix anchor placement for figures with captions.
\usepackage{hypcap}% it must be loaded after hyperref.
% Set up styles of URL: it should be placed after hyperref.
\urlstyle{same}

\addto\captionsenglish{\renewcommand{\contentsname}{Table of Contents}}

\usepackage{sphinxmessages}
\setcounter{tocdepth}{0}

\usepackage{svg}

\title{RGVFlood User Inteface Requirements Determination}
\date{Jan 10, 2022}
\release{0.1.1}
\author{Andrew N.S. Ernest, Ph.D., P.E., BCEE, D.WRE \and Christopher B. Fuller, Ph.D. \and William Kirkey, Ph.D. \and Peter Kirkey, \and Linda Navarro, \and Ivan Santos-Chavez, \and Carlos Reyes}
\newcommand{\sphinxlogo}{\sphinxincludegraphics{RATESLogo.png}\par}
\renewcommand{\releasename}{Release}
\makeindex
\begin{document}

\pagestyle{empty}
\sphinxmaketitle
\pagestyle{plain}
\sphinxtableofcontents
\pagestyle{normal}
\phantomsection\label{\detokenize{requirements/index::doc}}



\chapter{Introduction}
\label{\detokenize{requirements/introduction/index:introduction}}\label{\detokenize{requirements/introduction/index::doc}}
\sphinxAtStartPar
User Interface Requirements Determination refers to delineation of the process to be followed to implement any enhancements necessary to \DUrole{xref,std,std-term}{REON.cc} to accommodate the goals of \DUrole{xref,std,std-term}{REON/WM}. The initial deployment of the \DUrole{xref,std,std-term}{REON/WM} is to support \DUrole{xref,std,std-term}{RGVFlood}, partially funded by the \DUrole{xref,std,std-term}{TWDB}/FIF program.


\section{Need \& Philosophical Basis for REON.cc}
\label{\detokenize{requirements/introduction/need:need-philosophical-basis-for-reon-cc}}\label{\detokenize{requirements/introduction/need::doc}}
\sphinxAtStartPar
\DUrole{xref,std,std-term}{REON.cc} is the user interface for \DUrole{xref,std,std-term}{REON}. \DUrole{xref,std,std-term}{REON}, a flagship program of \DUrole{xref,std,std-term}{RATES}, is dedicated to the core philosophy of:

\sphinxAtStartPar
\sphinxstyleemphasis{Democratizing Water Intelligence for Knowledge\sphinxhyphen{}Enabled Policy \& Decision Making}

\begin{figure}[htbp]
\centering
\capstart

\noindent\sphinxincludegraphics{{dikw}.png}
\caption{Data, Information, Knowledge, Wisdom pyramid espoused by the informatics discipline. Source: \sphinxurl{https://www.pngegg.com/en/png-mvanj}}\label{\detokenize{requirements/introduction/need:id1}}\end{figure}

\sphinxAtStartPar
As adopted by \DUrole{xref,std,std-term}{REON}, \DUrole{xref,std,std-term}{DIKW} refers to:
\begin{itemize}
\item {} 
\sphinxAtStartPar
\sphinxstylestrong{Data}: Addressing the monitoring needs of under\sphinxhyphen{}served areas to ensure technology and monitoring solutions are available to all.

\item {} 
\sphinxAtStartPar
\sphinxstylestrong{Information}: Translating water \& environmental data into actionable intelligence.

\item {} 
\sphinxAtStartPar
\sphinxstylestrong{Knowledge}: Educating decision makers and elected officials to promote knowledge\sphinxhyphen{}based decision making.

\item {} 
\sphinxAtStartPar
\sphinxstylestrong{Wisdom}: Supporting implementation through facilitation of collaborative efforts between stakeholders such as municipalities, academic institutions, not\sphinxhyphen{}for\sphinxhyphen{}profits, conservancy \& environmental groups as well as state and federal regulatory agencies.

\end{itemize}
\begin{itemize}
\item {} 
\sphinxAtStartPar
\sphinxstylestrong{Data}: Addressing the monitoring needs of under\sphinxhyphen{}served areas to ensure technology and monitoring solutions are available to all.

\item {} 
\sphinxAtStartPar
\sphinxstylestrong{Information}: Translating water \& environmental data into actionable intelligence.

\item {} 
\sphinxAtStartPar
\sphinxstylestrong{Knowledge}: Educating decision makers and elected officials to promote knowledge\sphinxhyphen{}based decision making.

\item {} 
\sphinxAtStartPar
\sphinxstylestrong{Wisdom}: Supporting implementation through facilitation of collaborative efforts between stakeholders such as municipalities, academic institutions, not\sphinxhyphen{}for\sphinxhyphen{}profits, conservancy \& environmental groups as well as state and federal regulatory agencies.

\end{itemize}


\section{The Need for a Common Operating Picture}
\label{\detokenize{requirements/introduction/cop:the-need-for-a-common-operating-picture}}\label{\detokenize{requirements/introduction/cop::doc}}\DUrole{admonition-six-blind-men-and-the-elephant}{}\DUrole{admonition-six-blind-men-and-the-elephant}{}
\begin{figure}[htbp]
\centering
\capstart

\noindent\sphinxincludegraphics[height=500\sphinxpxdimen]{{Blind_men_and_elephant4}.jpg}
\caption{Depiction of the old Indian \sphinxstylestrong{fable of the Six Blind Men and the Elephant}.
Source: \sphinxurl{https://commons.wikimedia.org/wiki/File:Blind\_men\_and\_elephant4.jpg}}\label{\detokenize{requirements/introduction/cop:id2}}\end{figure}

\sphinxAtStartPar
Apropos to the issues of developing of water policy or collaborative decision
making, each stakeholder or decision maker views the
water “elephant” from their own \sphinxstylestrong{parochial perspective}, and argues that
theirs is reality. Being able to provide a \sphinxstylestrong{Common Operating Picture} is
critical if multi\sphinxhyphen{}jurisdictional decision making is to be both effective and
sustainable. \DUrole{xref,std,std-term}{COP} has been adopted by several agencies including the \DUrole{xref,std,std-term}{DHS} and
\DUrole{xref,std,std-term}{NOAA}’s \DUrole{xref,std,std-term}{NWS}. \DUrole{xref,std,std-term}{COP} is a Situational Awareness capability
designed to ensure that
all participating stakeholder and/or \sphinxstylestrong{decision makers possess} the full suite
of \sphinxstylestrong{information necessary} for the job at hand.


\chapter{Goals \& Objectives of the User Interface}
\label{\detokenize{requirements/goals/index:goals-objectives-of-the-user-interface}}\label{\detokenize{requirements/goals/index::doc}}
\sphinxAtStartPar
The overarching goal of the REON.cc cyber\sphinxhyphen{}collaboratory is to serve as the \sphinxstylestrong{primary end\sphinxhyphen{}user interface} for all aspects of \sphinxstylestrong{model execution} and \sphinxstylestrong{data analysis}. Specific objectives of REON.cc include:
\begin{itemize}
\item {} 
\sphinxAtStartPar
Serve as the \sphinxstylestrong{primary user Interface} for all REON related activities.

\item {} 
\sphinxAtStartPar
Provide \sphinxstylestrong{access} to all available \sphinxstylestrong{data} relevant to the REON goals and needs of the REON user base.

\item {} 
\sphinxAtStartPar
Provide access to the \sphinxstylestrong{analytics}, \sphinxstylestrong{visualization} and \sphinxstylestrong{forecasting} tools developed to serve the needs of the user base.

\item {} 
\sphinxAtStartPar
Serve as the platform to encourage \sphinxstylestrong{engagement} between users and facilitate \sphinxstylestrong{multi}\sphinxhyphen{}\sphinxstylestrong{lateral decision making}.

\end{itemize}
\DUrole{admonition-objectives}{}

\chapter{Description of Development Process}
\label{\detokenize{requirements/process/index:description-of-development-process}}\label{\detokenize{requirements/process/index::doc}}
\sphinxAtStartPar
The continued development of the REON.cc User interface will follow the
\sphinxstylestrong{standard software development process}:
\begin{itemize}
\item {} 
\sphinxAtStartPar
\sphinxstylestrong{Requirements Determination}: Determination of the requirements for the next version of the User Interface.

\item {} 
\sphinxAtStartPar
\sphinxstylestrong{Predevelopment Plan}: Establishment of a plan for initiation the development process.

\item {} 
\sphinxAtStartPar
\sphinxstylestrong{Requirements Validation}: Validation of the User Interface Requirements by stakeholders and sponsor.

\item {} 
\sphinxAtStartPar
\sphinxstylestrong{Implementation}: Development \& implementation of the User Interface.

\item {} 
\sphinxAtStartPar
\sphinxstylestrong{Quality Assurance Demonstration}: Demonstration of operational integrity of the interface components and identification of methodologies for capturing un\sphinxhyphen{}intended outcomes. Equivalent to Alpha\sphinxhyphen{}Testing.

\item {} 
\sphinxAtStartPar
\sphinxstylestrong{End\sphinxhyphen{}User Acceptance Testing}: Rigorous testing of the beta version of the User Interface by stakeholders, with immediate feedback for refinement as needed.

\item {} 
\sphinxAtStartPar
\sphinxstylestrong{End\sphinxhyphen{}User Interface Development Report}: Production and provision of a final report documenting the User Interface development and implementation.

\end{itemize}


\chapter{Platform Identification}
\label{\detokenize{requirements/platform/index:platform-identification}}\label{\detokenize{requirements/platform/index::doc}}
\sphinxAtStartPar
The current implementation of REON.cc (Figure: REON.cc User Interface) utilizes a nominally modified instance of the \sphinxstylestrong{Open Source} Geospatial Content Management System GeoNode. From the provider:

\sphinxAtStartPar
\sphinxstyleemphasis{GeoNode is a web\sphinxhyphen{}based application and platform for developing geospatial information systems (GIS) and for deploying spatial data infrastructures (SDI).}

\sphinxAtStartPar
\sphinxstyleemphasis{It is designed to be extended and modified, and can be integrated into existing platforms.}

\sphinxAtStartPar
With a \sphinxstylestrong{robust user\sphinxhyphen{}management system}, integration with the well\sphinxhyphen{}established GeoServer \sphinxstylestrong{geospatial data server} platform for hosting GIS data, and reliance on the Django web framework for the Python programming language, GeoNode isideally suited to being adapted for addressing the Goals and Objectives of the REON.cc platform.

\begin{figure}[htbp]
\centering
\capstart

\noindent\sphinxincludegraphics{{user_interface}.jpg}
\caption{REON.cc User Interface}\label{\detokenize{requirements/platform/index:id1}}\end{figure}
\DUrole{admonition-reon-cc-user-interface}{}

\chapter{End\sphinxhyphen{}User Needs Identification}
\label{\detokenize{requirements/needs/index:end-user-needs-identification}}\label{\detokenize{requirements/needs/index::doc}}
\sphinxAtStartPar
Based on feedback during initial end\sphinxhyphen{}user meetings, the Figure: REON.cc Information Flow depicts the information flow necessary to meet the end\sphinxhyphen{}user needs. End\sphinxhyphen{}user needs are, in order of priority:
\begin{enumerate}
\sphinxsetlistlabels{\arabic}{enumi}{enumii}{}{.}%
\item {} 
\sphinxAtStartPar
\sphinxstylestrong{Hydrologic Extremes}: Immediate needs for REON.cc are to support policy and operational decision making for \sphinxstylestrong{flood response and resiliency}. Shortly after flood management application, user is expected to need inclusion of \sphinxstylestrong{drought preparedness and management} as core functionality.

\end{enumerate}
\begin{enumerate}
\sphinxsetlistlabels{\arabic}{enumi}{enumii}{}{.}%
\setcounter{enumi}{1}
\item {} 
\sphinxAtStartPar
\sphinxstylestrong{Water Supply}: As support is provided for floods and droughts, decision makers are expected to turn their attention to utilizing REON.cc for water supply applications, including mid\sphinxhyphen{}term \sphinxstylestrong{storage predictions} and \sphinxstylestrong{inter\sphinxhyphen{}basin transfer} decisions.

\item {} 
\sphinxAtStartPar
\sphinxstylestrong{Ecosystem Services}: Increased expansion of water supply needs are coupled to \sphinxstylestrong{water quality} implications on \sphinxstylestrong{source water}. In parallel, applications of REON.cc are expected to expand to address ecosystem services, including \sphinxstylestrong{Freshwater} and \sphinxstylestrong{instream flows} management.

\end{enumerate}

\begin{figure}[htbp]
\centering
\capstart

\noindent\sphinxincludegraphics{{information_flow}.jpg}
\caption{REON.cc Information Flow}\label{\detokenize{requirements/needs/index:id1}}\end{figure}
\DUrole{admonition-information-flow}{}

\chapter{Data}
\label{\detokenize{requirements/data/index:data}}\label{\detokenize{requirements/data/index::doc}}
\sphinxAtStartPar
Data is the lowest, most fundamental, tier of the DIKW pyramid. Data Input needs include processes for recruiting data specific to the REON.cc end\sphinxhyphen{}uses. This includes user entered, warehoused and cloud stored data.


\section{Crowdsource Data}
\label{\detokenize{requirements/data/crowdsource:crowdsource-data}}\label{\detokenize{requirements/data/crowdsource::doc}}
\sphinxAtStartPar
Effective data input will rely on a \sphinxstylestrong{balance} between \sphinxstylestrong{quality assured} data from known sources, to \sphinxstylestrong{crowdsourced} data, that while potentially meeting only nominal quality assurance standards, may often be of \sphinxstylestrong{higher volumes} and potentially possess \sphinxstylestrong{unique intrinsic value}.
\begin{itemize}
\item {} 
\sphinxAtStartPar
Crowdsource data can consume \sphinxstylestrong{significant storage capacity}.

\item {} 
\sphinxAtStartPar
For example, \sphinxstylestrong{photographs} from \sphinxstylestrong{social media} platforms, \sphinxstylestrong{geolocated} and \sphinxstylestrong{time\sphinxhyphen{}stamped} have been shown to provide potentially \sphinxstylestrong{quantitative} insights into flood inundation depths to \sphinxstylestrong{correlate} against \sphinxstylestrong{forecasting} tools.

\end{itemize}


\section{Observed Data}
\label{\detokenize{requirements/data/observed:observed-data}}\label{\detokenize{requirements/data/observed::doc}}
\sphinxAtStartPar
Of particular interest is RTHS and \sphinxstylestrong{real\sphinxhyphen{}time data} from other providers, along with other observations such as \sphinxstylestrong{physical grab samples}, \sphinxstylestrong{meteorological measurements} and others that could serve as \sphinxstylestrong{forcing data} for mechanistic models, analytics and decision support tools.
\begin{itemize}
\item {} 
\sphinxAtStartPar
The current \sphinxstylestrong{RTHS database} is CUAHSI ODM 1.1.1 compliant. Additional data is stored in different databases to accommodate the needs of the source data structures.

\item {} 
\sphinxAtStartPar
Where possible \sphinxstylestrong{cloud\sphinxhyphen{}sourced} data should be retrieved \sphinxstylestrong{on demand} rather than being stored locally in order to \sphinxstylestrong{ensure currency}. \sphinxstylestrong{Caching} and \sphinxstylestrong{buffering} technologies should be used to prevent data access \sphinxstylestrong{latency} when needed for processing by analytics, visualization and forecast tools.

\end{itemize}


\section{Domain Data}
\label{\detokenize{requirements/data/domain:domain-data}}\label{\detokenize{requirements/data/domain::doc}}
\sphinxAtStartPar
This includes \sphinxstylestrong{geographic}, \sphinxstylestrong{hydrographic}, \sphinxstylestrong{socio\sphinxhyphen{}economic},
\sphinxstylestrong{socio\sphinxhyphen{}political} and other data that defines the \sphinxstylestrong{geospatial extent} of the
application domain.
\begin{itemize}
\item {} 
\sphinxAtStartPar
\sphinxstylestrong{National} and \sphinxstylestrong{local data} sources such as NLDAS can be \sphinxstylestrong{ingested} with existing \sphinxstylestrong{Metadata} specifications, while topographic survey results, such as those generated during the placement of RTHS stations will required specification of \sphinxstylestrong{quality assurance standards} and other Metadata requirements as additional data inputs.

\item {} 
\sphinxAtStartPar
As with cloud\sphinxhyphen{}sourced observed data, \sphinxstylestrong{cloud\sphinxhyphen{}sourced domain data} should be retrieved \sphinxstylestrong{on demand} rather than being stored locally in order to \sphinxstylestrong{ensure currency} and \sphinxstylestrong{minimize local storage} needs through \sphinxstylestrong{caching}.

\end{itemize}


\section{Processed Data}
\label{\detokenize{requirements/data/processed:processed-data}}\label{\detokenize{requirements/data/processed::doc}}
\sphinxAtStartPar
Processed Data is produced when \sphinxstylestrong{raw data} is \sphinxstylestrong{transformed} through
\sphinxstylestrong{analytic} or \sphinxstylestrong{deterministic} tools.
\begin{itemize}
\item {} 
\sphinxAtStartPar
Unless carefully produced, processed Data can contribute to extremely \sphinxstylestrong{high storage} volume needs.

\item {} 
\sphinxAtStartPar
\sphinxstylestrong{Balancing} the \sphinxstylestrong{computational effort} needed to produce the processed data against \sphinxstylestrong{storage volumes} needed to store it is critical to \sphinxstylestrong{efficiency}.

\item {} 
\sphinxAtStartPar
\sphinxstylestrong{Tuning} the \sphinxstylestrong{analytic} and \sphinxstylestrong{deterministic} tools to limit the production of processed data to only that needed to serve the decision making end can minimize both processing and storage needs.

\item {} 
\sphinxAtStartPar
In some cases, \sphinxstylestrong{high computational} effort and \sphinxstylestrong{high processed storage volumes} cannot be avoided, in which case, processed may be \sphinxstylestrong{condensed} through further analytic processing to reduce the volume of storage needed while \sphinxstylestrong{maintaining} the \sphinxstylestrong{intrinsic value} of the processing.

\item {} 
\sphinxAtStartPar
\sphinxstylestrong{Caching} should be used to reduce the \sphinxstylestrong{volume} for \sphinxstylestrong{storage} over \sphinxstylestrong{time}.

\end{itemize}


\chapter{Information}
\label{\detokenize{requirements/information/index:information}}\label{\detokenize{requirements/information/index::doc}}
\sphinxAtStartPar
\sphinxstylestrong{Information} is a key product of REON.cc. It results from the transformation of raw data into actionable intelligence.
\begin{itemize}
\item {} 
\sphinxAtStartPar
Processing capacity needs are driven by the complexity and volume of data needed for the particular analytic and visualization tool being used, but in general, they are significantly lower that the mechanistic tools used in Knowledge: Processing Needs.

\item {} 
\sphinxAtStartPar
Storage needs for information are driven by the outputs of the analytic and procedural tools that act on the raw data.

\item {} 
\sphinxAtStartPar
Determination of the need to store these outputs is driven by a balance between the computation effort required to reproduce and the volume of storage associated with the output data.

\item {} 
\sphinxAtStartPar
Some caching and buffering must be considered for remote and cloud\sphinxhyphen{}served data, such as National data sources, to ensure timely utilization in the Processing elements of REON.cc.

\item {} 
\sphinxAtStartPar
The REON.cc cyberinfrastructure has been implemented for water quality applications (Navarro et al, 2021).

\item {} 
\sphinxAtStartPar
Tools involved in transformation of Date into Information include:

\end{itemize}


\section{Analytics}
\label{\detokenize{requirements/information/analytics:analytics}}\label{\detokenize{requirements/information/analytics::doc}}
\sphinxAtStartPar
Most analytic tools are unlikely to require output data storage, as the \sphinxstylestrong{computational effort} needed to process the tool are likely to be \sphinxstylestrong{nominal}, and as such can be executed \sphinxstylestrong{on\sphinxhyphen{}demand}, allowing for outputs reflective of the current state of available data.
\begin{itemize}
\item {} 
\sphinxAtStartPar
For \sphinxstylestrong{example}, a simple analytic tool that takes stage height data from hydrologically connected RTHS stations, and produces travel time estimates between the two, produces a single data point for each execution.

\item {} 
\sphinxAtStartPar
Both the computational \sphinxstylestrong{cost} and the storage cost are \sphinxstylestrong{nominal}.

\item {} 
\sphinxAtStartPar
However, travel time \sphinxstylestrong{estimates} for two RTHS stations can \sphinxstylestrong{change over time} as the bathymetry of the intervening channel evolves, and under deferring upstream and downstream flow conditions.

\item {} 
\sphinxAtStartPar
\sphinxstylestrong{Storage} of the different travel time estimates \sphinxstylestrong{can be useful over time} to provide insights into the changing topography of the region.

\end{itemize}


\section{Visualization}
\label{\detokenize{requirements/information/visualization:visualization}}\label{\detokenize{requirements/information/visualization::doc}}
\sphinxAtStartPar
Visualization tools can potentially generate \sphinxstylestrong{large volumes} of produced data, ranging from static \sphinxstylestrong{imagery} to \sphinxstylestrong{videos}.
\begin{itemize}
\item {} 
\sphinxAtStartPar
Visualization tools can also be \sphinxstylestrong{off\sphinxhyphen{}loaded} for \sphinxstylestrong{client\sphinxhyphen{}side execution}, in which case server\sphinxhyphen{}side storage of outputs is not applicable.

\item {} 
\sphinxAtStartPar
Visualization is a \sphinxstylestrong{core tool} for establishing COP between \sphinxstylestrong{collaborating decision makers}, and is anticipated to be critical component of the operational functionality of REON.cc.

\item {} 
\sphinxAtStartPar
Except in the case of large volumes of source data, the \sphinxstylestrong{computational effort} required to produce the visualization outputs is generally \sphinxstylestrong{nominal}.

\item {} 
\sphinxAtStartPar
A \sphinxstylestrong{balance} between ensuring \sphinxstylestrong{consistency} in source data between runs, and minimizing \sphinxstylestrong{latency} between requests for visuals and their production, and minimizing storage volumes must be drawn.

\item {} 
\sphinxAtStartPar
The \sphinxstylestrong{likely approach} will be to establish visualization “\sphinxstylestrong{instances}” that are \sphinxstylestrong{cached} for a prescribed period to ensure all participating entities have access to the COP.

\end{itemize}


\chapter{Knowledge}
\label{\detokenize{requirements/knowledge/index:knowledge}}\label{\detokenize{requirements/knowledge/index::doc}}
\sphinxAtStartPar
\sphinxstylestrong{Knowledge} refers to the \sphinxstylestrong{transformation} of Data and Information through a fundamental understanding of the \sphinxstylestrong{physical}, \sphinxstylestrong{chemical} and \sphinxstylestrong{biological} mechanisms into \sphinxstylestrong{Actionable Intelligence}.
\begin{itemize}
\item {} 
\sphinxAtStartPar
These mechanisms are \sphinxstylestrong{represented} as \sphinxstylestrong{mathematical} or \sphinxstylestrong{statistical transformations}, with the resulting outputs being \sphinxstylestrong{reproducible}

\item {} 
\sphinxAtStartPar
Several \sphinxstylestrong{processing components} are integrated into REON.cc:

\end{itemize}


\section{Assimilation}
\label{\detokenize{requirements/knowledge/assimilation:assimilation}}\label{\detokenize{requirements/knowledge/assimilation::doc}}
\sphinxAtStartPar
The Assimilation tools are used import or transform Observed Data into the required format for ingestion into Analytics or other Forecasting tools. The key application is for “nudging” deterministic tools using timely forcing data.
\begin{itemize}
\item {} 
\sphinxAtStartPar
They include routines that retrieve and ingest domain and forcing data for the mechanistic forecasting tools.

\item {} 
\sphinxAtStartPar
Processing capacity needs can be significant, and can increase exponentially with the scale of the domain

\item {} 
\sphinxAtStartPar
Storage needs can be orders of magnitude greater than the input Data and Information used, especially with time\sphinxhyphen{}variant computations.

\item {} 
\sphinxAtStartPar
Reduction of output storage needs can be significantly reduced by transforming it through analytic tools.

\end{itemize}


\section{Forecasting}
\label{\detokenize{requirements/knowledge/forecasting:forecasting}}\label{\detokenize{requirements/knowledge/forecasting::doc}}
\sphinxAtStartPar
Forecasting tools are differentiated from Analytics primarily in scope and
rigor. Both types of tools can be either Stochastic or Mechanistic, however
forecasting tools are more likely to be both Mechanistic and Deterministic.

\sphinxAtStartPar
Within the context of REON.cc, the immediate application of forecasting tools will fall under the following categories:


\subsection{Hydrologic Modeling}
\label{\detokenize{requirements/knowledge/hydrologic:hydrologic-modeling}}\label{\detokenize{requirements/knowledge/hydrologic::doc}}
\sphinxAtStartPar
Hydrologic modeling focuses on simulation of water quantity in the environment. Some key elements of hydrologic modeling include:
\begin{itemize}
\item {} 
\sphinxAtStartPar
Conservation of Mass: A focus on balancing the amount, or mass, of water in the system.

\item {} 
\sphinxAtStartPar
Stochastic Processes: Hydrologic processes are often represented as Stochastic.

\item {} 
\sphinxAtStartPar
Land Surface Models: LSM’s are critical in rigorous hydrologic forecasting, since forecasts can be significantly influenced by meteorological processes.

\item {} 
\sphinxAtStartPar
Groundwater Models: Groundwater Models are necessary components if significant interaction is expected between surface water and groundwater. Groundwater timescales are often orders of magnitude higher than surface water. Exceptions such as Karst formations will often require coupled Surface and Groundwater modeling.

\item {} 
\sphinxAtStartPar
Hydrologic Routing: Hydrologic routing of surface water is necessary to translate land surface Models into lateral transport through defined channels. While grid\sphinxhyphen{}based land surface Models will allow for lateral transport between cells, simulation of channelized flow is necessary under most topographic conditions where channelized flow dominates.

\item {} 
\sphinxAtStartPar
Hydrologic Models: Several hydrologic models were considered for initial incorporation into the REON/WM, including:
\begin{itemize}
\item {} 
\sphinxAtStartPar
WRF\sphinxhyphen{}Hydro

\item {} 
\sphinxAtStartPar
HEC\sphinxhyphen{}HMS

\item {} 
\sphinxAtStartPar
VIC

\end{itemize}

\end{itemize}

\sphinxAtStartPar
\sphinxstylestrong{TIER I Real\sphinxhyphen{}Time Hydrologic Model}: WRF\sphinxhyphen{}Hydro was selected for the initial deployment within the REON/WM system for Tier I modeling because:
\begin{itemize}
\item {} 
\sphinxAtStartPar
It is an open\sphinxhyphen{}source, community\sphinxhyphen{}supported framework, allowing for extensibility.

\item {} 
\sphinxAtStartPar
It is what the NWC uses for the NWM.

\item {} 
\sphinxAtStartPar
Facilitates a pathway for contribution to, and the adoption of, the Nextgen\sphinxhyphen{}NWM as it matures.

\end{itemize}
\DUrole{admonition-tier-i-real-time-hydrologic-model}{}

\subsection{Hydraulic Modeling}
\label{\detokenize{requirements/knowledge/hydraulic:hydraulic-modeling}}\label{\detokenize{requirements/knowledge/hydraulic::doc}}
\sphinxAtStartPar
Hydraulic modeling is typically differentiated from Hydrologic Modeling in that formulations balance system energy rather than mass. This is typically required in situations where the timescales of transport and transformation phenomena are short.

\sphinxAtStartPar
Some key elements of hydrologic modeling include:
\begin{itemize}
\item {} 
\sphinxAtStartPar
Conservation of Energy: A focus on balancing energy in the system.

\item {} 
\sphinxAtStartPar
Mechanistic Processes: Hydraulic systems are often represented as combinations of Mechanistic processes.

\item {} 
\sphinxAtStartPar
Hydraulic Models: Several hydraulic models were considered for initial incorporation into the REON/WM, including:
\begin{itemize}
\item {} 
\sphinxAtStartPar
HEC\sphinxhyphen{}RAS

\item {} 
\sphinxAtStartPar
SPRNT

\item {} 
\sphinxAtStartPar
Primo

\end{itemize}

\end{itemize}

\sphinxAtStartPar
\sphinxstylestrong{Tier II On\sphinxhyphen{}Demand Hydraulic Modeling}: HEC\sphinxhyphen{}RAS was selected for initial deployment in the REON/WM as the Tier II model because:
\begin{itemize}
\item {} 
\sphinxAtStartPar
It is extensively used by practitioners

\item {} 
\sphinxAtStartPar
Though closed\sphinxhyphen{}source, it is freely available for use by end\sphinxhyphen{}users.

\item {} 
\sphinxAtStartPar
It enjoys strong and stable development support by the USACE/HEC

\end{itemize}
\DUrole{admonition-tier-ii-on-demand-hydraulic-modeling}{}

\subsection{Urban Stormwater Modeling}
\label{\detokenize{requirements/knowledge/stormwater:urban-stormwater-modeling}}\label{\detokenize{requirements/knowledge/stormwater::doc}}\begin{quote}

\sphinxAtStartPar
Urban stormwater modeling typically includes facets of both Hydrologic Modeling and Hydraulic Modeling, and will often incorporate rudimentary water quality modeling as well. These models simulate rainfall\sphinxhyphen{}runoff, open channel flow and underground pipe flow to try to capture as much of the urban stormwater environment as possible. These are complex models requiring a significant amount of domain data to be effective and useful.

\sphinxAtStartPar
\sphinxstylestrong{Tier III Urban Stormwater Modeling}: Due to the complexity of the data requirements, urban stormwater modeling is not expected to be directly linked or embedded into REON.cc.
\end{quote}


\chapter{Wisdom}
\label{\detokenize{requirements/wisdom/index:wisdom}}\label{\detokenize{requirements/wisdom/index::doc}}
\sphinxAtStartPar
Collaborative decision making is a foundational precept of the REON concept,
requiring an integration between scientific principles and policy drivers
(Gutenson et al, 2020).


\section{Decision Support}
\label{\detokenize{requirements/wisdom/dss:decision-support}}\label{\detokenize{requirements/wisdom/dss::doc}}
\sphinxAtStartPar
Decision Support Tools are generally combinations Analytics and Visualization,
sometimes with inference logic built in, that guide the end\sphinxhyphen{}user through
decision logic.
\begin{itemize}
\item {} 
\sphinxAtStartPar
Early Warning

\item {} 
\sphinxAtStartPar
Inundation Depth

\item {} 
\sphinxAtStartPar
Damage Assessment

\item {} 
\sphinxAtStartPar
Alternatives Analysis

\end{itemize}


\section{User Collaboration}
\label{\detokenize{requirements/wisdom/collaboration:user-collaboration}}\label{\detokenize{requirements/wisdom/collaboration::doc}}
\sphinxAtStartPar
User collaboration is predication on establishment of a COP, ensuring
discussions and collaborative decision making can be made from a common
reference point.
\begin{itemize}
\item {} 
\sphinxAtStartPar
Data \& Information Sharing

\item {} 
\sphinxAtStartPar
Trans\sphinxhyphen{}Jurisdictional Knowledge Inventory

\end{itemize}


\chapter{Workflow Implications}
\label{\detokenize{requirements/workflow/index:workflow-implications}}\label{\detokenize{requirements/workflow/index::doc}}
\sphinxAtStartPar
The immediate goals for the REON.cc user Interface are to support regional flood
planning and collaborative decision making. REON.cc must:
\begin{itemize}
\item {} 
\sphinxAtStartPar
Assimilate available, timely and necessary Data to support all the flood
planning and decision making functions of REON.cc.

\item {} 
\sphinxAtStartPar
Provide key Analytics and Visualization tools to interpret the Data or drive
further investigations

\item {} 
\sphinxAtStartPar
Transform the Data and Information into actionable intelligence to support
operational decisions and policy making.

\item {} 
\sphinxAtStartPar
Serve as a platform for collaborative, multi\sphinxhyphen{}jurisdictional decision making.

\end{itemize}


\chapter{Approach}
\label{\detokenize{requirements/approach/index:approach}}\label{\detokenize{requirements/approach/index::doc}}\begin{itemize}
\item {} 
\sphinxAtStartPar
\sphinxstylestrong{GeoNode}: The primary user interface platform will rely on the GeoNode
open\sphinxhyphen{}source geospatial content management system. GeoNode will be extended
with web applications to provide the tools and services described herein.

\item {} 
\sphinxAtStartPar
\sphinxstylestrong{Data Ingestion}: Server\sphinxhyphen{}side scripts will be developed to automate
ingestion of Data to maintain currency.

\item {} 
\sphinxAtStartPar
\sphinxstylestrong{WRF\sphinxhyphen{}Hydro}: The WRF\sphinxhyphen{}Hydro hydrologic forecast model will run continuously in
the background. The primary interaction between the WRF\sphinxhyphen{}Hydro instance and
the REON.cc interface will be for visualization of forecast data. The long
term goal of the hydrologic forecasting tool development process is to be
model agnostic, allowing for ingestion of forecast data from a variety of
platforms, including the ability to provide ensemble forecasts.

\end{itemize}
\begin{itemize}
\item {} 
\sphinxAtStartPar
\sphinxstylestrong{HEC\sphinxhyphen{}RAS}: Scripts will be developed to extract and transform Data from the
WRF\sphinxhyphen{}Hydro model to be used in HEC\sphinxhyphen{}RAS hydraulic models. Elements of this
process will include extraction of domain data, including topographic and
hydraulic survey results, along with hydrologic forecasts, all clipped to
the target domain. The HEC\sphinxhyphen{}RAS model will be run on the client\sphinxhyphen{}side
computer, utilizing the analytics and visualization tools available locally,
however, a long\sphinxhyphen{}term goal will be to provide up\sphinxhyphen{}load capability for model
scenarios and outputs to REON.cc for contribution to the knowledge base and
potential inclusion in server\sphinxhyphen{}side analytics and decision support tools.

\end{itemize}
\begin{itemize}
\item {} 
\sphinxAtStartPar
\sphinxstylestrong{SWMM}: Similar to HEC\sphinxhyphen{}RAS, scripts will be developed to extract and transform
Data from the WRF\sphinxhyphen{}Hydro model to be used in the SWMM urban stormwater
models. Elements of this process will include extraction of domain data,
including topographic and hydraulic survey results, along with hydrologic
forecasts, all clipped to the target domain. The SWMM model will be run on
the client\sphinxhyphen{}side computer, utilizing the analytics and visualization tools
available locally. Initial deployments will required the client\sphinxhyphen{}side
deployments to provide detailed delineation of the urban stormwater network
not defined in the existing REON.cc repositories. Future goals will include
ingestion of the urban stormwater networks for automated generation of the
input data for client\sphinxhyphen{}side execution, and potentially server\sphinxhyphen{}side execution
as well.

\end{itemize}
\begin{itemize}
\item {} 
\sphinxAtStartPar
\sphinxstylestrong{Dashboard}: The web\sphinxhyphen{}portal for REON.cc will provide a dashboard,
providing information tuned to the users’ specific interests and needs. The
dashboard will also allow for execution of analytics, visualization and
decision support tools as they are added to the platform.

\item {} 
\sphinxAtStartPar
\sphinxstylestrong{Notifications}: Critical to promoting user collaboration will be the
integration of a notification and communication system both between users
and with processes, including the triggering of early warnings based on the
RTHS network and forecast tools.

\end{itemize}


\chapter{Indices and tables}
\label{\detokenize{requirements/index:indices-and-tables}}\begin{itemize}
\item {} 
\sphinxAtStartPar
\DUrole{xref,std,std-ref}{genindex}

\item {} 
\sphinxAtStartPar
\DUrole{xref,std,std-ref}{modindex}

\item {} 
\sphinxAtStartPar
\DUrole{xref,std,std-ref}{search}

\end{itemize}



\renewcommand{\indexname}{Index}
\printindex
\end{document}